%%  docs/README.tex - Introduction to the documentation.
%%
%%     Copyright (C) 2024-2025 Kıvılcım Defne Öztürk
%%
%% This program is free software: you can redistribute it and/or modify
%% it under the terms of the GNU General Public License as published by
%% the Free Software Foundation, either version 3 of the License, or
%% (at your option) any later version.
%%
%% This program is distributed in the hope that it will be useful,
%% but WITHOUT ANY WARRANTY; without even the implied warranty of
%% MERCHANTABILITY or FITNESS FOR A PARTICULAR PURPOSE.  See the
%% GNU General Public License for more details.
%%
%% You should have received a copy of the GNU General Public License
%% along with this program. If not, see <https://www.gnu.org/licenses/>.

\documentclass{book}
\usepackage[utf8]{inputenc}
\usepackage{hyperref}
\usepackage[a4paper, left=2.5cm, right=2.5cm, top=1in, bottom=1in]{geometry}
\usepackage{microtype}
\title{LAMENT: Lefty Application, Modification, Editing and Notification Tool}
\author{S. L. S. Sacramentum}
\date{08 April 2025}

\begin{document}
    \maketitle
    \pagenumbering{roman}
    \tableofcontents
    \pagenumbering{arabic}
    \chapter{The Documentation Preamble}
    \section{Introduction}
    This is the first booklet in the documentation of Lefty Framework, which covers the documentation for LAMENT.
    LAMENT is the reference implementation of Sacramentum's Configuration Theory. Configuration theory describes
    how a computer system is configured and how this configuration can be applied to fleets of machines in a
    uniform way. It builds upon the ideas presented in Eelco Dolstra's thesis \textbf{\textit{The Purely Functional Software Deployment Model}} and
    makes them more accessible to the end users.

    \textit{Configuration Theory} presents us how computer systems are configured, how changes are applied, and more importantly,
    \textit{how declarative and immutable configuration can live together with plain old mutable configuration}.
    \newpage
    \section{License}
    \newpage
    \section{Contributions to LAMENT}\label{contributions-to-lament}

\subsection{Preamble}\label{preamble}

Free and open source software development is a collaborative job.
Various people work on the source code, and different people have
different styles of programming. While we value diversity a lot here,
the presence of a unified coding style is essential for everyone to
provide a more comfortable and convenient environment to collaborate.
Therefore, decisions about the style of the code, the way the pull
requests are described, and the way pull requests are merged should be
standardised.

\subsection{General Rules}\label{general-rules}

\begin{itemize}
\tightlist
\item
  Abide by the \href{CODE_OF_CONDUCT.md}{Code of Conduct}. We are very
  strict about this one. We will not tolerate harassment of any
  contributors here.
\end{itemize}

\subsection{Coding style}\label{coding-style}

\begin{itemize}
\tightlist
\item
  We generally follow the
  \href{https://github.com/luarocks/lua-style-guide}{style guidelines of
  LuaRocks}.
\item
  However we have one strict enforcement, use \texttt{local\ function}s
  whenever possible.
\end{itemize}

\subsection{Pull requests}\label{pull-requests}

\begin{itemize}
\tightlist
\item
  Use concise titles for pull requests and describe what it does well.
\item
  (The applicability of this article to forks on external servers not
  guaranteed) GitHub gives us a nice feature called \emph{tags} on pull
  requests. Use the appropriate tags for your pull requests.
\item
  And that's it you will have a large chance for it being merged.
\end{itemize}

\subsection{Commits (for project
administrators)}\label{commits-for-project-administrators}

\begin{itemize}
\tightlist
\item
  \textbf{VERY IMPORTANT:} The only commits on the \texttt{main} branch
  should be created by merges, unless some sort of apocalypse happens.
\item
  \textbf{Always create branches} for your commits and make a
  \textbf{pull request} to the \texttt{main} branch.
\end{itemize}

\subsection{Final notes}\label{final-notes}

\begin{itemize}
\tightlist
\item
  Do not be afraid to ask whenever you need help. We have both
  \href{https://github.com/Sparkles-Laurel/lament/discussions/10}{GitHub
  Discussions} and a
  \href{https://matrix.to/\#/\#lament-contrib:platypus-sandbox.com}{Matrix
  room}
\end{itemize}

    \newpage
    \include{CODE_OF_CONDUCT}
    \newpage
    \chapter{Introduction}
    \include{chapters/INTRODUCTION}
    \chapter{Configuration as an Object}
    \include{chapters/CONFIGOM}
\end{document}
